\documentclass[11pt]{article}

% Edit this customize for an instructor

\newcommand{\instructorpronoun}[1]{his}

% Use this when displaying a new command

\newcommand{\command}[1]{``\lstinline{#1}''}
\newcommand{\program}[1]{\lstinline{#1}}
\newcommand{\url}[1]{\lstinline{#1}}
\newcommand{\channel}[1]{\lstinline{#1}}
\newcommand{\option}[1]{``{#1}''}
\newcommand{\step}[1]{``{#1}''}

\usepackage{pifont}
\newcommand{\checkmark}{\ding{51}}
\newcommand{\naughtmark}{\ding{55}}

\usepackage{listings}
\lstset{
  basicstyle=\small\ttfamily,
  columns=flexible,
  breaklines=true
}

% Define the headers and footers

\usepackage{fancyhdr}

\usepackage[margin=1in]{geometry}
\usepackage{fancyhdr}

\pagestyle{fancy}

\fancyhf{}
\rhead{Computer Science 203}
\lhead{Syllabus}
\rfoot{Page \thepage}
\lfoot{Spring 2020}

% Use elastic spacing around the headers

\usepackage{titlesec}
\titlespacing\section{0pt}{6pt plus 4pt minus 2pt}{4pt plus 2pt minus 2pt}

\newcommand{\syllabustitle}[1]
{
  \begin{center}
    \begin{center}
      \bf
      CMPSC 203\\Software Engineering\\
      Spring 2020\\
      \medskip
    \end{center}
    \bf
    #1
  \end{center}
}

\begin{document}

\thispagestyle{empty}

\syllabustitle{Syllabus}

\vspace*{-1em}
\subsection*{Course Instructor}
Dr.\ Gregory M.\ Kapfhammer\\
\noindent Office Location: Alden Hall 108 \\
\noindent Office Phone: +1 814--332--2880 \\
\noindent Email: \url{gkapfham@allegheny.edu} \\
\noindent Twitter: \url{@GregKapfhammer} \\
\noindent Web Site: \url{https://www.gregorykapfhammer.com/}

\subsection*{Instructor's Office Hours}

\begin{itemize}

  \itemsep.05em

  \item Monday: 11:00 am--12:00 noon and 2:30 pm--3:30 pm (15 minute time slots)

  \item Tuesday: 4:30 pm--5:00 pm (15 minute time slots)

  \item Wednesday: 11:00 am--12:00 noon and 2:30 pm--3:30 pm (15 minute time slots)

  \item Thursday: 4:30 pm--5:00 pm (15 minute time slots)

  \item Friday: 11:00 am--12:00 noon and 2:30 pm--3:30 pm (15 minute time slots)

\end{itemize}

\noindent To schedule a meeting with me during my office hours, please visit my
web site and click the ``About $\rightarrow$ Schedule'' link in the top
left-hand corner. Now, you can browse my office hours or schedule an appointment
by clicking the correct link and then reserving an open time slot. Students are
also encouraged to post appropriate questions to a channel in Slack, which is
available at \url{https://CMPSC203Spring2020.slack.com/}, and monitored by the
instructor and the teaching assistants.

\subsection*{Course Meeting Schedule}

Lecture, Discussion, and Group Work: Monday and Wednesday, 1:30 pm--2:20 pm \\
Practical Session: Friday, 1:30 pm--2:20 pm \\
Laboratory Session: Tuesday, 2:30 pm--4:20 pm \\
Final Examination: Friday, May 1, 2020 at 7:00 pm

\subsection*{Course Description}

\begin{quote}

A human-centric study of the principles used during the engineering of
high-quality software systems. In addition to examining the human behaviors and
social processes undergirding software development methodologies, students
participate in teams tasked with designing, developing, and delivering a
significant software application for a customer. During a weekly laboratory
session, students use state-of-the-art software engineering, management, and
communication tools to complete projects, reporting on their results through
both written documents and oral presentations. Prerequisite: CMPSC
101. Distribution Requirements: SB, SP.\@ \\

\end{quote}

\noindent Students can access the course's web site by visiting the instructor's
site, clicking the ``Courses'' link in the top left-hand corner, and looking for
the name and number of this course in the listing.

\subsection*{Course Objectives}

The process of developing software involves the application of a number of
interesting techniques and tools. During this class, we will explore the phases
of the software engineering lifecycle and examine the principles, challenges,
and open questions associated with each phase. Throughout the semester, we will
investigate the interplay between the theory and practice of software
engineering. Specifically, we will delve into the details of requirements
elicitation, design, implementation, testing, documentation, maintenance, and
deployment through a discussion of book chapters and articles from the software
engineering literature. Along with learning more about how to effectively work
in a team of diverse software developers, students will enhance their ability
to write and present ideas about software in a clear and compelling fashion.
Students will develop an understanding of the fascinating connections between
computer science and software engineering and other disciplines in the social
and natural sciences and the humanities. Students also will gain software
engineering experience when completing practical assignments and large-scale
projects.

\vspace*{-.25em}
\subsection*{Performance Objectives}

At the completion of this class, a student should be aware of the fundamental
challenges associated with software engineering. Furthermore, students should
be comfortable with a wide array of concepts, methodologies, techniques, and
tools that they apply to the problem of developing large software systems. A
successful student will emerge with more than an understanding of the tools
(e.g., text editors, compilers, linters, debuggers, automated testing
frameworks, integrated development environments, and version control systems)
that a software engineer uses. A student should also have an understanding of
the software engineering lifecycle and the activities that take place in each
of its phases. Finally, a student should have an understanding of some of the
current research and the open questions in the field of software engineering.
After completing this class, a student should be equipped for further graduate
study in the fields of computer science and software engineering. The student
should also be able to participate in real-world software development projects
by adeptly using cutting-edge software tools and working with a team of diverse
developers.

\vspace*{-.25em}
\subsection*{Required Textbooks}

\noindent{\em Cooperative Software Design}. Andrew J.\ Ko.
Online Edition, 14 chapters, 2018.
%
\vspace*{.25em}

\noindent{\em A Philosophy of Software Design}. John Ousterhout.
First Edition, 178 pages, 2018.
%
\vspace*{.25em}

\noindent{\em Think Python}. Allen B. Downey.
Second Edition or Online Edition, 292 pages, 2015.
%
\vspace*{.25em}

\noindent{\em Exercises in Programming Style}. Cristina Videira Lopes.
First Edition, 282 pages, 2014.
%
\vspace*{.25em}

\noindent{\em Python Testing with Pytest}. Brian Okken.
First Edition, 197 pages, 2017.
%
\vspace*{.5em}

\noindent
The instructor will also reference, when necessary, these optional books:
%
\vspace*{.25em}

\noindent{\em Software Engineering: Theory and Practice}. Shari Lawrence
Pfleeger and Joanne M. Atlee.\\ Fourth Edition, 792 pages, 2010.
%
\vspace*{.25em}

\noindent{\em The Mythical Man Month}. Frederick P.\ Brooks, Jr.
Second Edition, 336 pages, 1995.

\vspace*{-.25em}
\subsection*{Course Policies}

\subsubsection*{Grading}

The grade that a student receives in this class will be based on the following
categories. All of these percentages are approximate and, if the need to do so
presents itself, it is possible for the course instructor to change the assigned
percentages during the academic semester.

\renewcommand{\arraystretch}{1.2}

\begin{center}
  \begin{tabular}{ll}
    Class Participation        & 5\%  \\
    First Examination          & 10\% \\
    Second Examination         & 10\% \\
    Final Examination          & 15\% \\
    Practical Assignments      & 15\% \\
    Project Reports            & 15\% \\
    Software Projects          & 30\%
  \end{tabular}
\end{center}

\noindent
These grading categories have the following definitions:

\vspace*{-.05in}

\begin{itemize}

  \item {\em Class Participation\/}: All students are required to actively
    participate during all of the course sessions. Your participation will take
    forms such as answering questions about the reading assignments, asking
    constructive questions of group members, giving presentations, and leading a
    discussion. You also must regularly participate in the discussions in the
    course's Slack workspace. A student will receive an interim and final letter
    grade for this category.

  \item {\em First and Second Examinations\/}: The first and second interim
    examinations will cover all of the material in their associated module(s),
    as outlined on a review sheet. While the second examination is not
    cumulative, it will assume that a student has a basic understanding of the
    material that was the focus of the first examination. The date for the first
    and second examinations will be announced at least one week in advance of
    the scheduled date. Unless prior arrangements are made with the course
    instructor, all students will be expected to take these examinations on the
    scheduled date and complete the tests in the stated period of time.

  \item {\em Final Examination\/}: The final examination is a three-hour
    cumulative test. By enrolling in this course, students agree that, unless
    there are extenuating circumstances, they will take the final examination at
    the date and time stated on the first page of the syllabus.

  \item {\em Practical Assignments\/}: Graded on a credit/no-credit basis, these
    assignments allow students to enhance the technical skills that they learned
    in previous class and laboratory sessions.

  \item {\em Project Reports\/}: Graded on both a letter-grade and a
    credit/no-credit basis, these reports invite students to furnish evidence of
    their mastery of the technical and professional knowledge and skills in
    software engineering. Students will submit reports and receive feedback on
    ways in which they can improve their mastery of the aforementioned skills.
    At the completion of a project, students will submit a report documenting
    their overall knowledge and skills.

  \item {\em Software Projects\/}: Graded on a letter-grade basis, these
    team-based projects invite students to design, implement, test, document,
    deploy, and maintain a high-quality software system released under an
    open-source licence through a publicly available version control repository.
    In addition to encouraging students to enhance their technical knowledge and
    skills, these projects will invite you to refine your ability to work
    effectively with a team.

\end{itemize}

\subsubsection*{Assignment Submission and Evaluation}

All assignments will have a stated due date. Electronic versions of the
practical assignments and project reports must be submitted to a student's
GitHub repository; students will learn how to use version control with GitHub
during the first laboratory and practical sessions. Electronic versions of the
software projects must also be submitted through a designated public GitHub
repository. No credit will be awarded for any course work that is not submitted
to your GitHub repository with the required name.
%
Unless specified otherwise, all project reports must be turned in at the
beginning of the session that is one week after the day the assignment was
released. Practical assignments will be due at the start of the next class
session, unless otherwise stated. The two software projects will have deadlines
determined in consultation with the students in the course. Without making
special arrangements with the instructor, no work will be accepted after the
published deadline.
%
Using a report that the instructor shares with you through the commit log in
GitHub, you will privately received a grade for and feedback on each assignment.
Your grade will be a function of whether or you not completed correct work and
submitted it by the deadline. Other factors (e.g., the quality of your source
code, technical writing, and team work) will also influence your assignment's
grade.

\vspace*{-.1in}

\subsubsection*{Course Attendance}

It is mandatory for all students to attend all of the class, practical, and
laboratory sessions. If, due to extenuating circumstances, you will not be able
to attend a session, then, whenever possible, please see the course instructor
at least one week in advance to describe your situation. Students who miss more
than five unexcused sessions will have their final grade in the course reduced
by one letter grade. Students who miss more than ten of the aforementioned
events will fail the course.

\vspace*{-.05in}

\subsubsection*{Use of Laboratory Facilities}

Throughout the semester, we will employ many different software tools that
computer scientists use during the design, implementation, and evaluation of
computer software. Since it is unlikely that a student's unmodified computer
will have the software tools that correctly support the completion of the
course's assignments, unless there are documented extenuating circumstances,
students are advised to complete all of their work for this course while using
the laboratory facilities.

\vspace*{-.1in}

\subsubsection*{Class Preparation}

In order to minimize confusion and maximize learning, students must invest time
to prepare for the class discussions, lectures, and practical sessions. During
the class periods, the course instructor will often pose challenging questions
that could require group discussion, the creation of a program or data set, a
vote on a thought-provoking issue, or a group presentation. Only students who
have prepared for class by reading the assigned material and reviewing the
current project and practical assignments will be able to effectively
participate in these discussions about software engineering.

More importantly, only prepared students will be able to acquire the knowledge
and skills that they need to be successful in this course, subsequent courses,
and the field of software engineering. In order to help students remain
organized and to effectively prepare for classes, the course instructor will
maintain a class schedule with reading assignments and presentation slides.
During the class sessions students will also be required to download, use, and
modify programs and data sets that are made available through means such as the
course web site and a GitHub repository.

\subsubsection*{Seeking Assistance}

Students who are struggling to understand the knowledge and skills developed in
a class, laboratory, or practical session are encouraged to seek assistance from
the course instructor, the teaching assistants, or the departmental tutors.
Throughout the semester, students should, within the bounds of the Honor Code,
ask and answer questions on the Slack workspace for our course; please request
assistance from the instructor, teaching assistants, and tutors first through
Slack before sending an email. Students who need the course instructor's
assistance must schedule a meeting through \instructorpronoun{} web site and
come to the meeting with all of the details needed to discuss their question.

\subsubsection*{Using GitHub}

When completing the software projects, students must give evidence of their work
through GitHub's issue tracker and logs of pull requests and commits. Unless you
make prior arrangements with the instructor, only work submitted through GitHub
will count towards your grade for these projects.

\subsubsection*{Using Email}

Although we will primarily use Slack for non-GitHub communication, I will
sometimes use email to send announcements about important matters such as
schedule changes. It is your responsibility to check your email at least once a
day and to ensure that you can reliably send and receive emails. This class
policy is based on the statement about the use of email that appears in {\em The
Compass}, the College's student handbook; please see the instructor if you do
not have this handbook.

\subsubsection*{Honor Code}

The Academic Honor Program that governs the entire academic program at Allegheny
College is described in the Allegheny Academic Bulletin. The Honor Program
applies to all work that is submitted for academic credit or to meet non-credit
requirements for graduation at Allegheny College. This includes all work
assigned for this class (e.g., examinations, laboratory assignments, and the
final project). All students who have enrolled in the College will work under
the Honor Program. Each student who has matriculated at the College has
acknowledged the following pledge:

\vspace*{-.10in}
%
\begin{quote}
%
  I hereby recognize and pledge to fulfill my responsibilities, as defined in
  the Honor Code, and to maintain the integrity of both myself and the College
  community as a whole.
%
\end{quote}
%
\vspace*{-.10in}

\noindent It is understood that an important part of the learning process in any
course, and particularly one in software engineering, derives from thoughtful
discussions with teachers and fellow students. Such dialogue is encouraged.
However, it is necessary to distinguish carefully between the student who
discusses the principles underlying a problem with others and the student who
produces assignments that are identical to, or merely variations on, someone
else's work. While it is acceptable for students in this class to discuss their
programs, documents, and reports with their classmates, deliverables that are
nearly identical to the work of others will be taken as evidence of violating
the \mbox{Honor Code}.

\subsubsection*{Disability Services}

The Americans with Disabilities Act (ADA) is a federal anti-discrimination
statute that provides comprehensive civil rights protection for persons with
disabilities. Among other things, this legislation requires all students with
disabilities be guaranteed a learning environment that provides for reasonable
accommodation of their disabilities. Students with disabilities who believe they
may need accommodations in this class are encouraged to contact Disability
Services at 332--2898. Disability Services is part of the Learning Commons and
is located in Pelletier Library. Please do this as soon as possible to ensure
that approved accommodations are implemented in a timely fashion.

\subsection*{Welcome to an Adventure in Software Engineering}

In reference to software, Frederick Brooks, Jr.\ wrote in {\em The Mythical Man
Month}, ``The magic of myth and legend has come true in our time.'' Software is
a pervasive aspect of our society that changes how we think and act. High
quality software also has the potential to positively influence the lives of
many people. Moreover, the specification, design, implementation, testing,
maintenance, and documentation of software are exciting and rewarding
activities! At the start of this class, I invite you to pursue, with great
enthusiasm and vigor, this adventure in software engineering.

\end{document}
